\documentclass[10pt]{beamer}

\usetheme{metropolis}
% \usecolortheme{beaver}
% \usecolortheme{whale}

\usepackage{appendixnumberbeamer}

\usepackage{booktabs}
\usepackage[scale=2]{ccicons}

\usepackage{pgfplots}
\usepgfplotslibrary{dateplot}

\usepackage{xspace}
\newcommand{\themename}{\textbf{\textsc{metropolis}}\xspace}


\usepackage{amsmath}
\usepackage{amssymb}
\usepackage{multicol}
\usepackage{lipsum}
\usepackage{mathtools}
\usepackage{mathpartir}
\usepackage[mathscr]{eucal}

\usepackage{stmaryrd}

\usepackage{relsize}

\usepackage{listings}
% \usepackage{lstlangcoq}
\usepackage{tikz}
\usetikzlibrary{snakes}

\usepackage{pdfpages}
\usepackage{wasysym}

%%%%%%%%%%%%%%%%%%%%%%%%%%%
% common
\newcommand{\doublebrackets}[1]{[\![ #1 ]\!]}
\newcommand{\progspec}[1]{\{ #1 \}}
\newcommand{\tuple}[1]{\langle #1 \rangle}
\newcommand{\sep}{ ~|~ }
\newcommand{\triple}[3]{\progspec{ #1 } #2 \progspec{ #3 }}
\newcommand{\cross}{\diagdown\mkern-16mu\diagup}
% \newcommand{\cross}{$\bigtimes$}
% \newcommand{\VDash}{\shortparallel\mkern-8mu\vDash}
\DeclareRobustCommand{\VDash}{\mathrel{||}\joinrel\Relbar}
%%%%%%%%%%%%%%%%%%%%%%%%%%%

%%%%%%%%%%%%%%%%%%%%%%%%%%%
% iris cf lang
\newcommand{\true}{\texttt{true}}
\newcommand{\false}{\texttt{false}}
\newcommand{\cmdref}{\texttt{ref}}
\newcommand{\cmdfork}[1]{\texttt{fork}\{#1\}}
\newcommand{\cmdloop}[1]{\texttt{loop}_{#1}\,}
\newcommand{\cmdbreak}{\texttt{break}\,}
\newcommand{\cmdcontinue}{\texttt{continue}}
\newcommand{\cmdif}{\texttt{if}\,}
\newcommand{\cmdthen}{\,\texttt{then}\,}
\newcommand{\cmdelse}{\,\texttt{else}\,}
\newcommand{\cmdseq}{\,;;\,}
\newcommand{\cmdreturn}{\texttt{return}\,}
\newcommand{\cmdcall}{\texttt{call}\,}

\newcommand{\tunit}{\texttt{unit}}
\newcommand{\tint}{\texttt{int}}
\newcommand{\tbool}{\texttt{bool}}
\newcommand{\vtype}[1]{\textbf{V}[#1]}

\newcommand{\pure}[1]{\text{PenCtx}(#1)}
\newcommand{\wellf}[1]{\text{well\_formed}(#1)}

\newcommand{\hred}{\rightarrow_h}
\newcommand{\tred}{\rightarrow_t}
\newcommand{\tpred}{\rightarrow_{tp}}
\newcommand{\cred}{\text{red}}
%%%%%%%%%%%%%%%%%%%%%%%%%%%

%%%%%%%%%%%%%%%%%%%%%%%%%%%
% iris logic
\newcommand{\largeast}{\,|\mkern-10mu\diagdown\mkern-16mu\diagup}
\newcommand{\WP}{\textsf{WP}}
\newcommand{\wand}{-\mkern-8mu*\,}
\newcommand{\upd}{\dot{|\mkern-8mu\Rrightarrow}\,}
\newcommand{\later}{\triangleright}
\newcommand{\wpre}[5]{\WP\, #1\, \progspec{#2, [#3, #4, #5]}}
\newcommand{\htriple}[6]{\progspec{#1}\, #2\, \progspec{#3}\, \progspec{#4}\, \progspec{#5}\, \progspec{#6}}
%%%%%%%%%%%%%%%%%%%%%%%%%%%

%%%%%%%%%%%%%%%%%%%%%%%%%%%
% toy lang
\newcommand{\cdealloc}[1]{[#1]}
\newcommand{\cseq}{;;}
\newcommand{\cif}[3]{\texttt{if}~#1~\texttt{then}~#2~\texttt{else}~#3}
\newcommand{\cfor}[2]{\texttt{for}(;;#2)~#1}
\newcommand{\cbreak}{\texttt{break}}
\newcommand{\ccontinue}{\texttt{continue}}
\newcommand{\creturn}{\texttt{return}}
\newcommand{\cskip}{\texttt{skip}}
%%%%%%%%%%%%%%%%%%%%%%%%%%%

%%%%%%%%%%%%%%%%%%%%%%%%%%%
% big step
\newcommand{\bigvalid}{\vDash_\mathrm{b}}
\newcommand{\exitkind}{\mathrm{exit\_kind}}
\newcommand{\ek}{\mathrm{ek}}
\newcommand{\ekb}{\mathrm{brk}}
\newcommand{\ekc}{\mathrm{con}}
\newcommand{\ekr}{\mathrm{ret}}
\newcommand{\bigstep}[4]{(#1, #2) \Downarrow (#3, #4)}
%%%%%%%%%%%%%%%%%%%%%%%%%%%

%%%%%%%%%%%%%%%%%%%%%%%%%%%
% continuation
\newcommand{\guard}[2]{\mathrm{guard}(#1, #2)}
\newcommand{\kloop}[2]{\mathrm{KLoop}(#1, #2)}
\newcommand{\kloops}[3]{\mathrm{KLoop}_{#3}(#1, #2)}
\newcommand{\kseq}[1]{\mathrm{KSeq}(#1)}
%%%%%%%%%%%%%%%%%%%%%%%%%%%

%%%%%%%%%%%%%%%%%%%%%%%%%%%
% utility
\newcommand{\todo}[1]{\textcolor{red}{[TODO: #1]}}
\newcommand{\figref}[1]{Fig.~\ref{#1}}
%%%%%%%%%%%%%%%%%%%%%%%%%%%

\newtheorem{thm}{Theorem}
\newtheorem{defn}[thm]{Definition} 
\newtheorem{lem}[thm]{Lemma}

\newcommand{\blue}[1]{{\color[rgb]{0, 0, 1} #1}}


\title{Iris-CF}
\subtitle{Iris Program Logic for Control Flow}
\date{\today}
\author{Wang Zhongye}
% \institute{Center for modern beamer themes}
% \titlegraphic{\hfill\includegraphics[height=1.5cm]{logo.pdf}}

\begin{document}

\maketitle

\begin{frame}{Table of contents}
  \setbeamertemplate{section in toc}[sections numbered]
  \tableofcontents[hideallsubsections]
\end{frame}

\section{Introduction}

\begin{frame}
\frametitle{What is Iris?}

\

Iris \cite{jung2017iris} is a Higher-Order Concurrent Separation Logic Framework,
implemented and verified in the Coq proof assistant.

\

\begin{figure}[ht]
    \tikzstyle{startstop} = [rectangle, minimum width = 1.5cm, minimum height = 0.5cm, text centered, draw = black]
    \tikzstyle{arrow} = [thick, ->]
    
    \centering
    
    \begin{tikzpicture}[node distance = 1.5cm]

    \node(base)[startstop] {Iris Base Logic};
    \node(mid)[above=30] at (base) {};
    \node(lang)[startstop, left=30] at (mid) {Programming Language};
    \node(logic)[startstop, above=60] at (base) {Iris Program Logic};

    \draw[arrow](base) -- (logic);
    \draw[arrow](lang) -- (mid);
    
    \end{tikzpicture}
    \caption{Iris Framework}
    \label{fig:iris-framework}
\end{figure}

\end{frame}

\begin{frame}
\frametitle{Why Iris-CF?}

Iris does not support reasoning about control flows very well.

\

One possible way to enable this is through continuation \cite{timany2019mechanized}, i.e., use try-catch style control flow.


\

\onslide<2->{
Our approach is like the multi-post-condition judgement in VST \cite{VST,VST-Floyd}.
\begin{mathpar}
  \inferrule[Seq]{
    \vdash \triple{P}{e_1}{R, [R_\ekb, R_\ekc, R_\ekr]} \and
    \vdash \triple{R}{e_2}{Q, [R_\ekb, R_\ekc, R_\ekr]}
  }{
    \vdash \triple{P}{e_1 \cmdseq e_2}{Q, [R_\ekb, R_\ekc, R_\ekr]}
  }
\end{mathpar}
}

% \begin{figure}[h]
% $
% (\cmdif x == 1 \cmdthen \cmdcontinue) \cmdseq e
% $

% $\Downarrow$

% $
% \Big(\cmdif x == 1 \cmdthen \text{\blue{throw} $()$ \blue{to} } (\cmdif x == 1 \cmdthen \cdots) \cmdseq e\Big) \cmdseq e
% $
% \end{figure}

\end{frame}

\section{Programming Language}

\begin{frame}
\frametitle{Values}

$$
v \in \textit{Val} ::=\, () \sep z \sep \true \sep \false \sep l \sep \lambda x.e \sep \cdots
$$

\end{frame}

\begin{frame}

\only<1,2,5,6>{
\frametitle{Expressions}
$$
\begin{aligned}
    e \in \textit{Expr} ::=\, & v \sep x \sep e_1(e_2) \\
    \sep\, & \cmdref(e) \sep !e \sep e_1 \leftarrow e_2 \\
    \sep\, & \cmdfork{e} \sep e_1 \cmdseq e_2 \sep \cmdif e_1 \cmdthen e_2 \cmdelse e_3 \\
    \onslide<2->{\sep\, & \blue{\cmdloop{e} e \sep \cmdbreak e \sep \cmdcontinue} \\}
    \onslide<5->{\sep\, & \blue{\cmdcall e \sep \cmdreturn e} \sep \cdots \\}
\end{aligned}
$$
\only<6>{
$$
\begin{aligned}
    \textit{FACT}' &\triangleq \lambda f. \lambda n. (\cmdif n = 0 \cmdthen \cmdreturn 1) \cmdseq n \times (\cmdcall (f\, f\, (n - 1))) \\
    \textit{FACT} &\triangleq \textit{FACT}'\,\textit{FACT}'
\end{aligned}
$$
}

}

\only<3,4>{
\frametitle{Evaluation Context}
$$
\begin{aligned}
    K \in \textit{Ctx} ::=\, & [] \sep K(e) \sep v(K) \\
    \sep\, & \cmdref(K) \sep !K \sep K \leftarrow e \sep v \leftarrow K \\
    \sep\, & K \cmdseq e \sep \cmdif K \cmdthen e_2 \cmdelse e_3 \\
    \sep\, & \blue{\cmdloop{e} K \sep \cmdbreak K} \sep \cdots
\end{aligned}
$$

\onslide<4>{
$$
\begin{aligned}
  \cmdloop{e} e = \cmdloop{e}[&e] && \rightarrow & \cmdloop{e}[&e'] = \cmdloop{e} e' \\
  &e  && \rightarrow & &e'
\end{aligned}
$$
}
}

\end{frame}

\begin{frame}
\frametitle{Evaluation Context}

$$
\begin{aligned}
    K \in \textit{Ctx} ::=\, & [] \sep K(e) \sep v(K) \\
    \sep\, & \cmdref(K) \sep !K \sep K \leftarrow e \sep v \leftarrow K \\
    \sep\, & K \cmdseq e \sep \cmdif K \cmdthen e_2 \cmdelse e_3 \\
    \sep\, & \blue{\cmdloop{e} K \sep \cmdbreak K \sep \cmdcall K \sep \cmdreturn K} \sep \cdots
\end{aligned}
$$

\end{frame}

\begin{frame}
\frametitle{Small Step Semantics}

$$
\begin{array}{c}  
(e, \sigma) \hred (e', \sigma', \vec{e}_f) \\
\text{where $\sigma, \sigma' \in \mathrm{state}$ and $\vec{e}_f \in \mathrm{list} \textit{ Expr}$}
\end{array}
$$

\begin{mathpar}
  \inferrule*[]{
    (e, \sigma) \hred (e', \sigma', \vec{e}_f)
  }{
    (K[e], \sigma) \tred (K[e'], \sigma', \vec{e}_f)
  }
\end{mathpar}
$$
\text{Here $\tred$ is thread local reduction.}
$$

\end{frame}

\begin{frame}
\frametitle{Small Step Semantics: Loop}

$$
\begin{aligned}
    (\cmdloop{e} v, \sigma) &\hred (\cmdloop{e} e, \sigma, \epsilon) && \\
    (\cmdloop{e} (\cmdcontinue), \sigma) &\hred (\cmdloop{e} e, \sigma, \epsilon) &&  \\
    \onslide<2->{(\cmdloop{e} (\cmdbreak v), \sigma) &\hred (v, \sigma, \epsilon) && \\}
    \onslide<3->{
    (K[\cmdbreak v], \sigma) &\hred (\cmdbreak v, \sigma, \epsilon) && \text{if } K \in \pure{\cmdbreak\!} \\
    (K[\cmdcontinue], \sigma) &\hred (\cmdcontinue, \sigma, \epsilon) && \text{if } K \in \pure{\cmdcontinue}
    }
\end{aligned}
$$
\onslide<3->{
$$
\begin{aligned}
  \text{where } \pure{\cmdbreak\!} &= \pure{\cmdcontinue}\\
  &\triangleq \textit{Ctx}^1\backslash\cmdloop{e}[]\backslash\cmdcall[]
\end{aligned}
$$
}
\end{frame}

\begin{frame}
\frametitle{Small Step Semantics: Call}

$$
\begin{aligned}
    (\cmdcall v, \sigma) &\hred (v, \sigma, \epsilon) \\
    (\cmdcall (\cmdreturn v), \sigma) &\hred (v, \sigma, \epsilon) \\
    (K[\cmdreturn v], \sigma) &\hred (\cmdreturn v, \sigma, \epsilon) && \text{if } K \in \pure{\cmdreturn\!}
\end{aligned}
$$
$$
\begin{aligned}
  \text{where } \pure{\cmdreturn\!} \triangleq \textit{Ctx}^1\backslash\cmdcall[]
\end{aligned}
$$

\end{frame}

\section{Program Logic: Iris}

\begin{frame}
\frametitle{Weakest Precondition (Iris)}

\only<1,2>{
$$
\triple{P}{e}{v.\, Q(v)}
$$
$$
\Updownarrow
$$
$$
\only<1>{P \Rightarrow \WP\, e \{v.\, Q(v)\}}
\only<2>{\Box (P \wand \WP\, e \{v.\, Q(v)\})}
$$
}

\only<3>{
$$
\begin{aligned}
  \sigma \vDash \WP\, e \{\Phi\} &\text{ iff. } (e \in \textit{Val} \land \sigma \vDash \Phi(e))\\
  &\,\,\lor \bigg(e \notin \textit{Val} \land \mathrm{reducible}(e, \sigma) \\
  &\,\,\quad \land \forall e_2, \sigma_2.\, \big((e, \sigma) \tred (e_2, \sigma_2, \epsilon)\big) \Rightarrow \sigma_2 \vDash \WP\, e_2 \{\Phi\} \bigg)
\end{aligned}
$$
}

\end{frame}

\begin{frame}
\frametitle{Proof in Iris}

\begin{mathpar}
  \inferrule*[]{
    \onslide<2->{
      \inferrule*[]{
        \onslide<3->{ \blue{P \vdash \WP\,e\{\Phi'\}} }\\\\
        \onslide<3->{ \Phi'(v) \vdash \WP\,K[v]\{\Phi\} }
      % }
    }{
      \onslide<2->{P \vdash \WP\,e\{v.\,\WP\,K[v]\{\Phi\}\}}
    }
    }
    ~~
    \onslide<2->{\blue{\WP\,e\{v.\WP\, K[v] \{\Phi\}\} \vdash \WP\,K[e]\{\Phi\}}}
  }{
    P \vdash \WP\,K[e]\{\Phi\}
  }
\end{mathpar}

\end{frame}


\section{Program Logic: Iris-CF}

\begin{frame}
\frametitle{Weakest Precondition (Iris-CF)}

$$
\begin{aligned}
    \sigma \vDash \wpre{e}{\Phi_N}{\Phi_B}{\Phi_C}{\Phi_R} &\text{ iff. }
           (e \in \textit{Val} \land \sigma \vDash \Phi_N(e)) \\
    & \lor \blue{(\exists v \in \textit{Val}.\, e = \cmdbreak v \land \sigma \vDash \Phi_B(v))} \\
    & \lor \blue{(e = \cmdcontinue \land \sigma \vDash \Phi_C())} \\
    & \lor \blue{(\exists v \in \textit{Val}.\, e = \cmdreturn v \land \sigma \vDash \Phi_R(v))} \\
    & \lor \biggl(e \notin \text{terminals} \land \cred(e, \sigma) \\
    & \quad \land \forall e', \sigma'. \bigl((e, \sigma) \tred (e', \sigma', \epsilon)\bigr) \Rightarrow \\
    & \quad\quad \sigma' \vDash \wpre{e'}{\Phi_N}{\Phi_B}{\Phi_C}{\Phi_R} \biggr)
\end{aligned}
$$

\end{frame}

\begin{frame}
\frametitle{Proof Rules: Basics}

$$
  \begin{aligned}
      \vdash \triple{\Phi_B(v)}{&\cmdbreak v}{{\bot},[{\Phi_B},{\bot},{\bot}]} \\
      \vdash \triple{\Phi_C()}{&\cmdcontinue}{{\bot},[{\bot},{\Phi_C},{\bot}]} \\
      \vdash \triple{\Phi_R(v)}{&\cmdreturn v}{{\bot},[{\bot},{\bot},{\Phi_R}]}
  \end{aligned}
$$
$$
\Downarrow
$$
$$
  \begin{aligned}
    \Phi_B(v) &\vdash \wpre{\cmdbreak v}{\bot}{\Phi_B}{\bot}{\bot} \\
    \Phi_C() &\vdash \wpre{\cmdcontinue}{\bot}{\bot}{\Phi_C}{\bot} \\
    \Phi_R(v) &\vdash \wpre{\cmdreturn v}{\bot}{\bot}{\bot}{\Phi_R}
  \end{aligned}
$$

\end{frame}

\begin{frame}
\frametitle{Proof Rules: Loop \& Call}

\begin{mathpar}
  \inferrule*[]{
    \vdash \triple{I}{e}{I,[\Phi_B,I,\Phi_R]}
  }{
    \vdash \triple{I}{(\cfor{e}{})}{\Phi_B, [\bot, \bot, \Phi_R]}
  }
\end{mathpar}
$$
\onslide<2->{
\Downarrow
}
$$
$$
\onslide<2->{
  \begin{aligned}
      \Box(I \wand{}\, \wpre{e}{\_. I}{\Phi_B}{\_. I}{\Phi_R}) * I &\vdash \wpre{(\cmdloop{e}e)}{\Phi_B}{\bot}{\bot}{\Phi_R}
  \end{aligned}
}
$$


\onslide<3->
$$
\wpre{e}{\Phi}{\bot}{\bot}{\Phi} \vdash \wpre{\cmdcall e}{\Phi}{\bot}{\bot}{\bot}
$$

\end{frame}

\begin{frame}
\frametitle{Proof Rule: Bind}

$$
\WP\,e\{v.\WP\, K[v] \{\Phi\}\} \vdash \WP\,K[e]\{\Phi\}
$$
$$
\onslide<2->{
\Downarrow
}
$$
$$
\onslide<2->{
\only<2>{
\begin{aligned}
  \WP\,e\,& \progspec{v. \wpre{K[v]}{\Phi_N}{\Phi_B}{\Phi_C}{\Phi_R}} \\
                  & \progspec{v. K \in \pure{\cmdbreak\!} \land \Phi_B(v)} \\
                  & \progspec{v. K \in \pure{\cmdcontinue} \land \Phi_C(v)} \\
                  & \progspec{v. K \in \pure{\cmdreturn\!} \land \Phi_R(v)}
\end{aligned} \vdash \wpre{K[e]}{\Phi_N}{\Phi_B}{\Phi_C}{\Phi_R}
}
\onslide<3>{
\begin{aligned}
  \WP\,e\,& \progspec{v. \wpre{K[v]}{\Phi_N}{\Phi_B}{\Phi_C}{\Phi_R}} \\
                  & \progspec{v. \blue{K \in \pure{\cmdbreak\!}} \land \Phi_B(v)} \\
                  & \progspec{v. \blue{K \in \pure{\cmdcontinue}} \land \Phi_C(v)} \\
                  & \progspec{v. \blue{K \in \pure{\cmdreturn\!}} \land \Phi_R(v)}
\end{aligned} \vdash \wpre{K[e]}{\Phi_N}{\Phi_B}{\Phi_C}{\Phi_R}
}
}
$$

\onslide<3>{
  If \blue{blue} assertions are invalidated, we should use other rules to prove the consequence.
}

\end{frame}

\begin{frame}
\frametitle{Proof Rule: Sequence}

$$
\begin{aligned}
  \WP\,e\,& \progspec{v. \wpre{K[v]}{\Phi_N}{\Phi_B}{\Phi_C}{\Phi_R}} \\
                  & \progspec{v. K \in \pure{\cmdbreak\!} \land \Phi_B(v)} \\
                  & \progspec{v. K \in \pure{\cmdcontinue} \land \Phi_C(v)} \\
                  & \progspec{v. K \in \pure{\cmdreturn\!} \land \Phi_R(v)}
\end{aligned} \vdash \wpre{K[e]}{\Phi_N}{\Phi_B}{\Phi_C}{\Phi_R}
$$
$$
\Downarrow
$$
$$
\begin{aligned}
  \WP\,e_1\,& \progspec{\_.\wpre{e_2}{\Phi_N}{\Phi_B}{\Phi_C}{\Phi_R}} \\
                  & \progspec{\Phi_B}\,\progspec{\Phi_C}\,\progspec{\Phi_R}
\end{aligned} \vdash \wpre{e_1 \cmdseq e_2}{\Phi_N}{\Phi_B}{\Phi_C}{\Phi_R}
$$

\end{frame}

\section{Conclusion}

\begin{frame}
\frametitle{Contextual Local Reasoning}
  
\begin{mathpar}
\inferrule[Hoare-Bind]{
  \vdash \triple{P}{e}{\Psi} \\ \forall v.\, \triple{\Psi(v)}{K[e]}{\Phi}
}{
  \vdash \triple{P}{K[e]}{\Phi}
}
\end{mathpar}

\begin{center}
  Untangle the proof of $e$ and $K$!  
\end{center}
  
\end{frame}

\begin{frame}
\frametitle{Our Achievement}

\begin{itemize}
  \item A lambda calculus like language supporting control flow and contextual local reasoning.
  \item A program logic build on Iris and allowing contextual local reasoning about control flow.
\end{itemize}

\end{frame}


\begin{frame}
\frametitle{Compare with Continuation}

$$
\begin{aligned}
  (K\big[\text{\blue{call/cc}}(x.\,e)\big], \sigma) &\rightarrow (K\big[e[\mathrm{cont}(K)/x]\big], \sigma) \\
  (K[\text{\blue{throw} $v$ \blue{to} } \mathrm{cont}(K')], \sigma) &\rightarrow (K'[v], \sigma)
\end{aligned}
$$

\begin{mathpar}
  \inferrule[callcc-wp]{
    \WP\,K\big[e[\mathrm{cont}(K)/x]\big]\{\Phi\}
  }{
    \WP\,K\big[\text{\blue{call/cc}}(x.\,e)\big]\{\Phi\}
  }
\end{mathpar}

\end{frame}

\appendix

\begin{frame}{References}

  \bibliography{paper.bib}
  \bibliographystyle{abbrv}

\end{frame}

\begin{frame}[standout]
  Questions?
\end{frame}


\begin{frame}{Backup slides}



\end{frame}

\end{document}
