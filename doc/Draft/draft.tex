\documentclass{article}
\usepackage[UTF8]{ctex}
\usepackage{color,amsmath,amssymb,graphicx,fancyhdr,amsfonts,amsthm,algorithmic,verbatim,bbold}
\usepackage{algorithm,hyperref}
\usepackage{mkolar_definitions}
\usepackage{extpfeil}
\usepackage{mathpartir}
\usepackage{wasysym}
\usepackage[mathscr]{eucal}
\usepackage{multirow}
\usepackage{booktabs}
\usepackage{relsize}
\usepackage[left=2cm,top=2cm,right=2cm]{geometry}
\numberwithin{algorithm}{section}

\newcommand{\doublebrackets}[1]{[\![ #1 ]\!]}
\newcommand{\progspec}[1]{\{ #1 \}}
\newcommand{\tuple}[1]{\langle #1 \rangle}

\newcommand{\true}{\texttt{true}}
\newcommand{\false}{\texttt{false}}
\newcommand{\cref}{\texttt{ref}}
\newcommand{\cfork}[1]{\texttt{fork}\{#1\}}
\newcommand{\cloop}[1]{\texttt{loop}_{#1}\,}
\newcommand{\cbreak}{\texttt{break}\,}
\newcommand{\ccontinue}{\texttt{continue}}
\newcommand{\cif}{\texttt{if}\,}
\newcommand{\cthen}{\,\texttt{then}\,}
\newcommand{\celse}{\,\texttt{else}\,}
\newcommand{\cseq}{\,;;\,}
\newcommand{\creturn}{\texttt{return}\,}
\newcommand{\ccall}{\texttt{call}\,}

\newcommand{\tunit}{\texttt{unit}}
\newcommand{\tint}{\texttt{int}}
\newcommand{\tbool}{\texttt{bool}}
\newcommand{\vtype}[1]{\textbf{V}[#1]}

\newcommand{\pure}[1]{\text{pure}(#1)}
\newcommand{\wellf}[1]{\text{well\_formed}(#1)}

\newcommand{\hred}{\rightarrow_h}
\newcommand{\tred}{\rightarrow_t}
\newcommand{\tpred}{\rightarrow_{tp}}
\newcommand{\cred}{\text{red}}

\newcommand{\wand}{-\mkern-8mu*\,}
\newcommand{\upd}{\dot{|\mkern-8mu\Rrightarrow}\,}
\newcommand{\later}{\triangleright}
\newcommand{\wpre}[5]{\textsf{wp}\, #1\, \progspec{#2}\, \progspec{#3}\, \progspec{#4}\, \progspec{#5}}
\newcommand{\htriple}[6]{\progspec{#1}\, #2\, \progspec{#3}\, \progspec{#4}\, \progspec{#5}\, \progspec{#6}}

\newcommand{\sep}{\,|\,}

\begin{document}

\paragraph{The Programming Language}

$$
\begin{aligned}
    v \in \textit{Val} ::=\, & () \sep z \sep \true \sep \false \sep l \sep \lambda x.e \sep \cdots \\
    e \in \textit{Expr} ::=\, & v \sep x \sep e_1(e_2) \sep \cref(e) \sep !e \sep e_1 \leftarrow e_2 \sep \cfork{e} \sep e_1 \cseq e_2 \sep \cloop{e} e \\
                        \sep\, & \cif e_1 \cthen e_2 \celse e_3 \sep \cbreak e \sep \ccontinue \sep \ccall e \sep \creturn e \sep \cdots \\
    K \in \textit{Ctx} ::=\, & \bullet \sep K(e) \sep v(K) \sep \cref(K) \sep !K \sep K \leftarrow e \sep v \leftarrow K \sep K \cseq e \sep \cloop{e} K \\
                        \sep\, & \cif K \cthen e_2 \celse e_3 \sep \cbreak K \sep \ccall K \sep \creturn K \sep \cdots
\end{aligned}
$$

$$
\texttt{loop}\, e \triangleq \cloop{e} e
$$

$\cbreak\!$ and $\ccontinue$ can not penetrate $\cloop{e}\!$, and $\creturn\!$ can not penetrate $\ccall\!$.
% A problem, $(\lambda x. \creturn x)(1) \cseq 3$ should evaluate to 1 or 3? In other words, whether the lambda expression is a function or a normal expression?

\paragraph{Reductions}

\subparagraph{Head Reductions}

$$
\begin{aligned}
    (\cloop{e} (), \sigma) &\hred (\cloop{e} e, \sigma, \epsilon) && \\
    (\cloop{e} (\cbreak v), \sigma) &\hred (v, \sigma, \epsilon) && \\
    (\cloop{e} (\ccontinue), \sigma) &\hred (\cloop{e} e, \sigma, \epsilon) &&  \\
    (K[\cbreak v], \sigma) &\hred (\cbreak v, \sigma, \epsilon) && \text{if } K \neq \bullet \text{ and } K \in \pure{\cbreak\!} \\
    (K[\ccontinue], \sigma) &\hred (\ccontinue, \sigma, \epsilon) && \text{if } K \neq \bullet \text{ and } K \in \pure{\ccontinue} \\
    (\ccall v, \sigma) &\hred (v, \sigma, \epsilon) \\
    (\ccall (\creturn v), \sigma) &\hred (v, \sigma, \epsilon) \\
    (K[\creturn v], \sigma) &\hred (\creturn v, \sigma, \epsilon) && \text{if } K \neq \bullet \text{ and } K \in \pure{\creturn\!} \\
    (\cif \true \cthen e_2 \celse e_3, \sigma) &\hred (e_2, \sigma, \epsilon) && \\
    (\cif \false \cthen e_2 \celse e_3, \sigma) &\hred (e_3, \sigma, \epsilon) && \\
    (v \cseq e, \sigma) &\hred (e, \sigma, \epsilon) && \\
    (\cfork{e}, \sigma) &\hred ((), \sigma, e) && \text{if } \wellf{e} \\
    &\cdots
\end{aligned}
$$

where
$$
\begin{aligned}
    \pure{\cbreak\!} &\triangleq \textit{Cxt}\,\backslash\cloop{e}\backslash\ccall \\
    % \{K \sep \forall K_1, K_2. \text{ if } K = K_1[K_2] \text{ then } K_1 \neq \cloop{e} K_2 \text{ and } K_2 \in \pure{\cbreak\!}\} \\
    \pure{\ccontinue} &\triangleq \textit{Cxt}\,\backslash\cloop{e}\backslash\ccall \\
    \pure{\creturn\!} &\triangleq \textit{Cxt}\,\backslash\ccall \\
    % \{K \sep \forall K_1, K_2. \text{ if } K = K_1[K_2] \text{ then } K_1 \neq \cloop{e} K_2 \text{ and } K_2 \in \pure{\ccontinue}\} \\
    \wellf{e} &\text{ iff. } % TODO: definition of well-definedness
    \end{aligned}
$$

\subparagraph{Thread Local Reductions}

$$
\begin{aligned}
    (K[e], \sigma) &\tred (K[e'], \sigma, \vec{e}_f) && \text{if } (e, \sigma) \hred (e', \sigma, \vec{e}_f) \\
    % (\creturn v, \sigma) &\tred (v, \sigma, \epsilon)
\end{aligned}
$$

The thread-pool reduction ($\tpred$) is identical to the original one.

% There is a small problem that the expression need to terminate after \cbreak{} and \ccontinue{}, otherwise the evaluation gets stuck.

\paragraph{Weakest Precondition}

$$
\begin{aligned}
    \wpre{e}{\Phi_N}{\Phi_B}{\Phi_C}{\Phi_R} \triangleq& \quad
           (e \in \textit{Val} \land \upd \Phi_N(e)) \\
    & \lor (e = \cbreak v \land \upd \Phi_B(v)) \\
    & \lor (e = \ccontinue \land \upd \Phi_C()) \\
    & \lor (e = \creturn v \land \upd \Phi_R(v)) \\
    & \lor \biggl(\forall \sigma. \cred(e) \land S(\sigma) \wand \upd \bigl(\cred(e, \sigma) \\
    & \quad \land \later
        \forall e', \sigma', \vec{e}_f. \bigl((e, \sigma) \tred (e', \sigma', \vec{e}_f)\bigr) \wand \upd \\
    & \quad\quad\quad \bigl(S(\sigma') * \wpre{e'}{\Phi_N}{\Phi_B}{\Phi_C}{\Phi_R} * \varhexstar_{e'\in\vec{e}_f} \wpre{e'}{\top}{\bot}{\bot}{\top} \bigr)\bigr)\biggr)
\end{aligned}
$$

$$
\cred(e) \triangleq e \notin \textit{Val}\, \cup \{\cbreak v, \ccontinue, \creturn v\}
$$

% The adequacy theorem will have some problem w.r.t the \creturn statement.

\paragraph{Rules for weakest preconditions}

$$
\begin{aligned}
    \Phi_B(v) &\vdash \wpre{\cbreak v}{\bot}{\Phi_B}{\bot}{\bot} & \textsc{(wp-break)} \\
    \Phi_C() &\vdash \wpre{\ccontinue}{\bot}{\bot}{\Phi_C}{\bot} & \textsc{(wp-continue)} \\
    \Phi_R(v) &\vdash \wpre{\creturn v}{\bot}{\bot}{\bot}{\Phi_R} & \textsc{(wp-return)} \\
    \Box(I \wand{}\, \wpre{e}{\_. I}{\Phi_B}{\_. I}{\Phi_R}) * I &\vdash \wpre{(\cloop{e}e)}{\Phi_B}{\bot}{\bot}{\Phi_R} & \textsc{(wp-loop)} \\
    \begin{aligned}
        \texttt{wp}\,e\,& \progspec{v. \wpre{K[v]}{\Phi_N}{\Phi_B}{\Phi_C}{\Phi_R}} \\
                        & \progspec{v. K \in \pure{\cbreak\!} \land \Phi_B(v)} \\
                        & \progspec{v. K \in \pure{\ccontinue} \land \Phi_C(v)} \\
                        & \progspec{v. K \in \pure{\creturn\!} \land \Phi_R(v)}
    \end{aligned} &\vdash \wpre{K[e]}{\Phi_N}{\Phi_B}{\Phi_C}{\Phi_R} & \textsc{(wp-bind)} \\
    % \wpre{e}{v. \wpre{K[v]}{\Phi_N}{\Phi_B}{\Phi_C}}{\bot}{\bot} &\vdash \wpre{K[e]}{\Phi_N}{\Phi_B}{\Phi_C} & \textsc{(wp-bind-normal)} \\
    % K\in\pure{\cbreak\!} \land \wpre{e}{\bot}{\Phi_B}{\bot} &\vdash \wpre{K[e]}{\bot}{\Phi_B}{\bot} & \textsc{(wp-bind-break)} \\
    \begin{aligned}
        &(v = \true \wand \wpre{e_2}{\Phi_N}{\Phi_B}{\Phi_C}{\Phi_R}) \\
        \land &(v = \false \wand \wpre{e_3}{\Phi_N}{\Phi_B}{\Phi_C}{\Phi_R})
    \end{aligned} &\vdash \wpre{\cif v \cthen e_2 \celse e_3}{\Phi_N}{\Phi_B}{\Phi_C}{\Phi_R} & \textsc{(wp-if)} \\
    \wpre{e}{\Phi_N}{\bot}{\bot}{\Phi_R} &\vdash \wpre{\ccall e}{\Phi_N \lor \Phi_R}{\bot}{\bot}{\bot} & \textsc{(wp-call)}
\end{aligned}
$$

% Return conditions should be merged with normal conditions upon calling the function.

\paragraph{Typing}

$$
A, B \in \textit{Type} ::= \epsilon \sep \tunit \sep \tint \sep \tbool \sep A \rightarrow B \sep \cdots
$$

\subparagraph{Syntactic Typing}

$$
    \Gamma \vdash e: A_N, A_B, A_C, A_R
$$

\begin{mathpar}
    \inferrule*[left=type-break]{
        \Gamma \vdash e: A_N, A_B, A_C, A_R
    }{
        \Gamma \vdash \cbreak e: \epsilon, A_N + A_B, A_C, A_R
    }
    \and
    \inferrule*[left=type-continue]{}{
        \Gamma \vdash \ccontinue: \epsilon, \epsilon, \tunit, \epsilon
    }
    \and
    \inferrule*[left=type-return]{
        \Gamma \vdash e: A_N, A_B, A_C, A_R
    }{
        \Gamma \vdash \creturn e: \epsilon, A_B, A_C, A_N + A_R
    }
    \and
    \inferrule*[left=type-loop]{
        \Gamma \vdash e: \tunit, A_B, \tunit, A_R
    }{
        \Gamma \vdash \cloop{e} e: A_B, \epsilon, \epsilon, A_R
    }
    \and
    \inferrule*[left=type-apply]{
        \Gamma \vdash e_1: A_N \rightarrow (A_N', A_B', A_C', A_R') \and
        \Gamma \vdash e_2: A_N, A_B, A_C, A_R
    }{
        \Gamma \vdash e_1(e_2): A_N', A_B + A_B', A_C + A_C', A_R + A_R'
    }
\end{mathpar}

where

$$
\begin{aligned}
    A + B &\triangleq \begin{cases}
        A & \text{if } B = \epsilon \text{ or } A = B \\
        B & \text{if } A = \epsilon \text{ or } A = B \\
        \text{undefined} & \text{otherwise}
    \end{cases} \\
    A \rightarrow (B_1, B_2, \cdots) &\triangleq A \rightarrow B_1, A \rightarrow B_2, \cdots
\end{aligned}
$$

\subparagraph{Semantic Typing}

$$
\Gamma \vDash e: A_N, A_B, A_C, A_R \triangleq \Gamma \vDash \htriple{\text{True}}{e}{\vtype{A_N}}{\vtype{A_B}}{\vtype{A_C}}{\vtype{A_R}}
$$

\end{document}